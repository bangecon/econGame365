% Options for packages loaded elsewhere
\PassOptionsToPackage{unicode}{hyperref}
\PassOptionsToPackage{hyphens}{url}
%
\documentclass[
]{article}
\usepackage{amsmath,amssymb}
\usepackage{iftex}
\ifPDFTeX
  \usepackage[T1]{fontenc}
  \usepackage[utf8]{inputenc}
  \usepackage{textcomp} % provide euro and other symbols
\else % if luatex or xetex
  \usepackage{unicode-math} % this also loads fontspec
  \defaultfontfeatures{Scale=MatchLowercase}
  \defaultfontfeatures[\rmfamily]{Ligatures=TeX,Scale=1}
\fi
\usepackage{lmodern}
\ifPDFTeX\else
  % xetex/luatex font selection
\fi
% Use upquote if available, for straight quotes in verbatim environments
\IfFileExists{upquote.sty}{\usepackage{upquote}}{}
\IfFileExists{microtype.sty}{% use microtype if available
  \usepackage[]{microtype}
  \UseMicrotypeSet[protrusion]{basicmath} % disable protrusion for tt fonts
}{}
\makeatletter
\@ifundefined{KOMAClassName}{% if non-KOMA class
  \IfFileExists{parskip.sty}{%
    \usepackage{parskip}
  }{% else
    \setlength{\parindent}{0pt}
    \setlength{\parskip}{6pt plus 2pt minus 1pt}}
}{% if KOMA class
  \KOMAoptions{parskip=half}}
\makeatother
\usepackage{xcolor}
\usepackage[margin=1in]{geometry}
\usepackage{graphicx}
\makeatletter
\def\maxwidth{\ifdim\Gin@nat@width>\linewidth\linewidth\else\Gin@nat@width\fi}
\def\maxheight{\ifdim\Gin@nat@height>\textheight\textheight\else\Gin@nat@height\fi}
\makeatother
% Scale images if necessary, so that they will not overflow the page
% margins by default, and it is still possible to overwrite the defaults
% using explicit options in \includegraphics[width, height, ...]{}
\setkeys{Gin}{width=\maxwidth,height=\maxheight,keepaspectratio}
% Set default figure placement to htbp
\makeatletter
\def\fps@figure{htbp}
\makeatother
\setlength{\emergencystretch}{3em} % prevent overfull lines
\providecommand{\tightlist}{%
  \setlength{\itemsep}{0pt}\setlength{\parskip}{0pt}}
\setcounter{secnumdepth}{-\maxdimen} % remove section numbering
\newlength{\cslhangindent}
\setlength{\cslhangindent}{1.5em}
\newlength{\csllabelwidth}
\setlength{\csllabelwidth}{3em}
\newlength{\cslentryspacingunit} % times entry-spacing
\setlength{\cslentryspacingunit}{\parskip}
\newenvironment{CSLReferences}[2] % #1 hanging-ident, #2 entry spacing
 {% don't indent paragraphs
  \setlength{\parindent}{0pt}
  % turn on hanging indent if param 1 is 1
  \ifodd #1
  \let\oldpar\par
  \def\par{\hangindent=\cslhangindent\oldpar}
  \fi
  % set entry spacing
  \setlength{\parskip}{#2\cslentryspacingunit}
 }%
 {}
\usepackage{calc}
\newcommand{\CSLBlock}[1]{#1\hfill\break}
\newcommand{\CSLLeftMargin}[1]{\parbox[t]{\csllabelwidth}{#1}}
\newcommand{\CSLRightInline}[1]{\parbox[t]{\linewidth - \csllabelwidth}{#1}\break}
\newcommand{\CSLIndent}[1]{\hspace{\cslhangindent}#1}
\usepackage{setspace}\doublespacing
\ifLuaTeX
  \usepackage{selnolig}  % disable illegal ligatures
\fi
\IfFileExists{bookmark.sty}{\usepackage{bookmark}}{\usepackage{hyperref}}
\IfFileExists{xurl.sty}{\usepackage{xurl}}{} % add URL line breaks if available
\urlstyle{same}
\hypersetup{
  pdftitle={An Open-Source Application to Computerize Simple Market Efficiency Games},
  pdfauthor={James T. Bang},
  hidelinks,
  pdfcreator={LaTeX via pandoc}}

\title{An Open-Source Application to Computerize Simple Market
Efficiency Games}
\author{James T. Bang}
\date{28 Sep 2023}

\begin{document}
\maketitle
\begin{abstract}
In-class experiments provide instructors a powerful tool for helping
students learn and understand market principles in economics. Despite
the effectiveness of experiments, economics instructors remain slow to
adopt them in their pedagogy. One reason for this lag could be the
time-consuming process of collecting, tabulating, and presenting the
outcomes of the experiments. This paper introduces functions and
ShinyApps in R for fast, free, in-class tabulation of the results of
five in-class market simulation experiments for teaching economics.\\
\textbf{Keywords:} Experiments; technology; teaching\\
\textbf{JEL Code:} A22
\end{abstract}

\pagenumbering{gobble}
\newpage
\pagenumbering{arabic}
\setcounter{page}{1}

\hypertarget{introduction}{%
\section{Introduction}\label{introduction}}

The use of experiments as demonstrations of economic theory date back at
least as far as those conducted by Chamberlin
(\protect\hyperlink{ref-chamberlin_experimental_1948}{1948}) and Smith
(\protect\hyperlink{ref-smith_experimental_1962}{1962}). These studies
were designed primarily to collect evidence supporting or refuting
economic models of rational behavior in market settings. Holt
(\protect\hyperlink{ref-holt_industrial_1993}{1993}) summarizes this
literature. While experiments remain an important method for observing
behavior to test economic hypotheses, these experiments have also found
their way into pedagogy
(\protect\hyperlink{ref-deyoung_market_1993}{DeYoung 1993}).

Classroom experiments offer students and instructors a fun departure
from the usual ``chalk and talk'' of explaining economic models. In
addition to entertainment value, studies have shown experiments to
increase student learning in post-test assessments
(\protect\hyperlink{ref-emerson_comparing_2004}{Emerson and Taylor
2004}; \protect\hyperlink{ref-dickie_classroom_2006}{Dickie 2006}). I
should note, however, that not all studies conclude that all types of
gamification improves learning by a significant margin: Gremmen and
Potters (\protect\hyperlink{ref-gremmen_assessing_1997}{1997}) find a
positive effect of games on average, but the effect is not statistically
significant, while Dickie
(\protect\hyperlink{ref-dickie_classroom_2006}{2006}) finds that games
do significantly improve learning, but that attaching grade incentives
to the games do not contribute any additional benefit. Moreover, Stodder
(\protect\hyperlink{ref-stodder_experimental_1998}{1998}) expresses
concern that classroom games that penalize cooperation may teach and
reinforce unethical decision making.

Despite the potential learning and entertainment value of classroom
experiments, they remain relatively rare among the pedagogies economics
professors adopt in their classrooms
(\protect\hyperlink{ref-watts_little_2008}{Watts and Becker 2008}). Two
factors may drive some of the hesitancy among economics instructors to
implement classroom experiments. On the one hand, free resources, such
as those described in the survey of non-computerized games by Brauer and
Delemeester (\protect\hyperlink{ref-brauer_games_2001}{2001}), require
significant time investments to tabulate and summarize the results. On
the other hand, automated resources, especially those distributed by
textbook publishers, impose a financial cost on students or their
institutions that instructors feel rightly averse to asking
budget-constrained students or departments to foot the bill for.

Cheung (\protect\hyperlink{ref-cheung_using_2008}{2008}) helps to
overcome this barrier by building tools for collecting student responses
to in-class non-computerized experiments using mobile phones and
texting. This contribution furthers this by automating the process of
calculating and summarizing the results of the experiments. These
examples only require students to be able to access a Google Form via
their browser on their computer or mobile device, which, given the
ubiquity of mobile phones among students (sometimes as their only
personal computing device), sets a fairly reasonable bar for
accessibility. A secondary contribution is simplifying existing versions
of classic market equilibrium, entry, and duopoly games with the hope of
increasing the diversity of methods used by economics instructors in the
classroom.

On the instructor's end, I have created a free, downloadable package
called \texttt{econGame} for the \emph{R} open-source statistical
computing program. The tabulation programs run as either stand-alone
functions in the \emph{R} console, or for in-class demonstrations as
\emph{Shiny} apps that can open in a browser tab if desired. This allows
the instructor to present the results of the experiment almost instantly
after the students have submitted their responses.

\hypertarget{description-of-the-experiments}{%
\section{Description of the
Experiments}\label{description-of-the-experiments}}

The models I will describe in this paper encompass two well-established
market equilibrium games known to the economics education literature
(see, for example, Williams and Walker
(\protect\hyperlink{ref-williams_economic_1993}{1993}) and Brauer and
Delemeester (\protect\hyperlink{ref-brauer_games_2001}{2001})), namely
the following experiments:\\
1. the pit market trading game introduced by Holt
(\protect\hyperlink{ref-holt_classroom_1996}{1996});\\
2. a simplified version of the entry and exit game by Garratt
(\protect\hyperlink{ref-garratt_free_2000}{2000});\\
3. three games simulating different oligopoly models (Bertrand, Cournot,
and Stackelberg);\\
4. the public-good game ;\\
5. a pollution policy game that compares different policy
alternatives(\protect\hyperlink{ref-anderson2000}{Anderson and Stafford
2000});\\
6. a special-interest lobbying game based on a lottery auction;\\
7. two games demonstrating behavioral economics (an anchoring game
similar to Gelman and Glickman
(\protect\hyperlink{ref-gelman2000}{2000}) and an ultimatum game based
on Kahneman, Knetsch, and Thaler
(\protect\hyperlink{ref-kahneman1986}{1986})).\\
For games where students work in small groups (usually pairs), the
package includes functions to create random groups if desired as well as
options for allowing students to choose their own groups. The package
also includes a plot function to create graphical illustrations of the
results as well as shiny applications to calculate and present the
results in a browser.\\
In the examples the ``payoffs'' students receive can be awarded to the
students at the end of the games as ``extra credit'' points, or
instructors may choose to encourage students to play the games
strategically, but only ``for the love of the game.'' I briefly describe
the delivery of the games below.

\hypertarget{pit-trading-market}{%
\subsection{Pit Trading Market}\label{pit-trading-market}}

Holt (\protect\hyperlink{ref-holt_classroom_1996}{1996}) designed the
pit market trading game for class sizes between 10 and 25 and takes
about 40-50 minutes to explain the game, play a few rounds, and tabulate
the results after each round. This game is an excellent illustration of
supply and demand, competitive market equilibrium, consumer and producer
surplus, and efficiency. The functions presented in this paper speed up
the response-collection process to allow the experiment to work for
larger classes. It also speeds up the calculation of the equilibrium and
graphs the equilibrium.

Before class, the instructor prepares (1) a Google Sheet assigning a
random integer between 1 and 10 representing each student's value that
they place on the asset;\footnote{The first sheet consists of a single
  formula in a single cell: ``=roundup(10*rand())''. A template can be
  found at:\\
  \url{https://docs.google.com/spreadsheets/d/1lCmC692ajsQZoatWtgZh5QKaJ9y3pOMt15JwRFaHanU/edit\#gid=258904023}.}
and (2) a Google Form through which students enter their bid and ask
prices;\footnote{A template can be found at:\\
  \url{https://docs.google.com/forms/d/1S_F9UJ6GXttxPqDLtk8Hg0ZgzDaHMxBmc1qH3W2gKZo/edit}.}
For the best compatibility with the result-tabulating function, users
who create their own forms should use the question prompts ``First
Name,'' ``Last Name'', ``Round,'' ``Value,'' ``Bid,'' and ``Ask.'' Add
text fields to insert additional context, instructions, or question
text.

In class, the instructor informs the students that they own a single
unit of an asset that each of them values differently. This value could
represent a profit that they can derive from using the asset as a
resource to produce other goods, a return the students expect to receive
from selling the asset in the future, or a subjective ``utility'' that
the students derive from using the asset as consumption. Students
discover this value by visiting a link to the first Google Sheet that
the instructor prepared to assign a random value from 1 to 10.

Students submit their name, the round number (if playing more than one),
their randomly-assigned value draw, a ``bid'' corresponding to the
highest amount they would pay for a second unit of the asset, and an
``ask'' corresponding to the lowest amount they would accept to part
with the unit of the asset they already own.

If the instructor decides to incentivize the game with points, students
keep their consumer and producer surpluses from each round as ``extra
credit'' points. \texttt{equilibriumGame} tabulates the supply and
demand schedules; calculates the equilibrium; graphs the equilibrium;
and tabulates the scores for each student.\footnote{The solution the
  piece-wise constant supply and demand equilibrium uses the help of a
  C++ helper function provided by ``David'' on Stack Overflow,\\
  \url{https://stackoverflow.com/questions/23830906/intersection-of-two-step-functions}.}

\hypertarget{entry-and-exit}{%
\subsection{Entry and Exit}\label{entry-and-exit}}

Garratt (\protect\hyperlink{ref-garratt_free_2000}{2000}) designed an
entry and exit game with four discrete specifications of the demand
functions for class sizes between 25 and 44. Garratt's version also
includes four markets (corn, wheat, rice, and soybeans), while the one
presented here only includes two (corn and soybeans). The demand
functions in this version of the experiment automatically adjust
according to the number of students participating. Garratt's version of
the game also takes about 45 minutes to complete the experiment
(including a government ``fallow program'' intervention variation),
usually about five rounds.

Before class the instructor prepares a Form to collect responses that
includes the fields ``First Name,'' ``Last Name,'' ``Round,'' and
``Market.''\footnote{A template can be found at:\\
  \url{https://docs.google.com/forms/d/1oUsLulfD5bqT6_9VVYIzLWuuQ-L4vwmC4jI-1jabOVQ/edit}.}
Other information that might be useful to add to the Form includes
information about demand and costs in each sector. The default inverse
demand functions allow for each market to reward producers with about
one unit of ``normal profit,'' but these settings can be changed.

In class, the instructor informs the students that they will choose to
plant corn, soybeans, or nothing. Producing corn incurs a cost of four,
while producing soybeans incurs a cost of 10. Garratt
(\protect\hyperlink{ref-garratt_free_2000}{2000}) recommends that the
instructor \emph{not} reveal the demand functions to students, whereas
some instructors (including me) might prefer to allow students to play
with perfect information. Selling a unit of corn brings revenue equal to
\(P_c = (N/2) + 6 - Q_c\), where \(N\) equals the number of students
participating and \(Q_c\) equals the number of students choosing to
produce corn. Selling a unit of soybeans brings revenue equal to
\(P_s = (N/2) + 10 - Q_s\). These parameters allow for there to be a
``normal profit'' of about one unit per student in each market in
equilibrium, to compensate for the risk of venturing into
self-employment.\footnote{The long run equilibrium, with 1 unit of
  ``normal profit'' occurs with \((N/2) + 1\) students choosing corn and
  \((N/2) - 1\) students choosing soybeans.} If the instructor wants the
prices to equal whole numbers (and the profits to equalize), they can
join the game as a ``student'' to round out the numbers.

If the instructor decides to play the game with points, students earn
points equal to their profits. Students may play as many rounds as the
instructor decides to continue the game, or until the markets reach the
long run equilibrium of zero \emph{economic} profit. Usually the markets
converge to the long run equilibrium by the end of about five rounds.
Padding the demand functions to leave one point of ``normal profit''
compared to sitting out simulates the concept of a normal profit
business owners receive for taking risk and lessens the chances that
students might ``win'' negative extra credit points. Students choosing
to produce nothing sell their labor in the labor market and earn zero
(they do not earn a normal profit).

\hypertarget{oligoply}{%
\subsection{Oligoply}\label{oligoply}}

I also constructed a set of games to demonstrate and compare equilibria
in different (two-firm) oligopoly models. In each of the examples,
students work in pairs. The instructor informs the students that the
market price depends on both the strategy they choose for their ``firm''
and also the strategy their partner chooses. Each of the three examples
uses the following linear inverse demand function (the parameters of
which individual instructors may change in the options):
\(P = a + b(Q_1 + Q_2),\) where the default values for the parameters
are \(a = 10\) and \(b = -1\). Likewise, firms face the the same cost
function: \(TC = f + cQ_i,\) where \(f\) represents the fixed cost (0 by
default) and \(c\) represents the (constant) marginal (and average) cost
of each additional unit (6 by default).

Before beginning any of the duopoly models, the instructor should solve
the competitive and monopoly equilibria with students first so that
students can see the plausible range of prices they should expect to
declare. Skipping this step often leads to a few greedy (but
quantitatively-challenged) students choosing prices that would result in
negative quantities. With the default parameters, the competitive
equilibrium price and quantity are 6 and 4, while the monopoly
equilibrium price and quantity are 8 and 2.

\hypertarget{bertrand-duoploy}{%
\subsubsection{Bertrand Duoploy}\label{bertrand-duoploy}}

Before class, the instructor prepares the Form to collect the responses,
which includes the fields ``First Name,'' ``Last Name'', ``Partner First
Name,'' ``Partner Last Name,'' ``Round,'' and ``Price.''\footnote{A
  template can be found at:\\
  \url{https://docs.google.com/forms/d/1AykOoY6mVj17D_5CW7-BLhSgOJdGEhyfYHKHROnvdcg/edit}.}
The package also includes a function to assign partners randomly using
the class roster (saved as a Google Sheet), and the
\texttt{bertrandGame()} function even allows the instructor to randomize
the partners \emph{after the fact} in case the instructor really wants
to cut down on tacit collusion. The package default calculates the
results using student-entered partners.

In class, the instructor reviews the competitive and monopoly equilibria
for the demand function in the example. The instructor then presents the
``rules'' for the Bertrand model as a ``winner takes all'' market.
Students submit their own names, partner's names, and their price. \[
Q_1 = \begin{cases}
    0 & \text{if } P_1 > P_2 \\ 
    (10 - P_1)/2 & \text{if } P_1 = P_2 \\
    10 - P_1 & \text{if } P_1 < P_2 
\end{cases}
\] If the instructor chooses to use points for the activity, students
earn points equal to their profit.

\hypertarget{cournot-duopoly}{%
\subsubsection{Cournot Duopoly}\label{cournot-duopoly}}

Before class, the instructor prepares the Form to collect the responses,
which includes the fields ``First Name,'' ``Last Name'', ``Partner First
Name,'' ``Partner Last Name,'' ``Round,'' and ``Strategy''\footnote{A
  template can be found at:\\
  \url{https://docs.google.com/forms/d/1dp-tUv5rNhRpm9UjFCy_pgsD4rJJnja-QJnWnMu81DI/edit}.}
As with the Bertrand game, instructors have discretion over allowing
students to choose their own partners (``rivals'') or randomizing the
partners before or after the students choose a strategy.

In the Cournot game, students choose either to ``collude'' (produce a
low quantity) or ``defect'' (produce a high quantity). The function that
tabulates the results assigns half of the monopolist's profit-maximizing
quantity to students who choose ``collude,'' and automatically assigns
the quantity corresponding to the best response function for students
who choose ``defect'' (based on the output choice of their rival).
Students only need to make the simple binary choice. Instructors using
this example for upper-level classes may (or may not) want to edit the
game settings to require students to submit a specific quantity (derive
the best response functions themselves).

\hypertarget{stackelberg-duopoly}{%
\subsubsection{Stackelberg Duopoly}\label{stackelberg-duopoly}}

Similar to the Cournot game, students in the Stackelberg game choose to
``collude'' or ``defect.'' In contrast to the Cournot game, students
must know their partner in advance, and followers will see the leaders'
strategy choices before choosing their strategy. The function again
automatically calculates the quantities corresponding to the set of
binary strategy choices to determine the payoff outcomes.\footnote{A
  template can be found at:\\
  \url{https://docs.google.com/forms/d/1vERPMPt_kW96JPAY6mEtkQMu6FLCgPuqoFL8i8bulYk/edit}.}

\hypertarget{public-good-game}{%
\subsection{Public Good Game}\label{public-good-game}}

Holt and Laury (\protect\hyperlink{ref-holt1997}{1997}) and Leuthold
(\protect\hyperlink{ref-leuthold1993}{1993}) conduct experiments
involving voluntary contributions to a public good. This example
somewhat resembles Leuthold's example more closely, but differs in that
(1) instructors can choose to denominate contributions in terms of
points instead of ``hypothetical'' monetary endowments; and (2) the
portion of points (or candy or hypothetical dollars) held for private
returns pay zero return (for simplicity).

Before class, the instructor prepares a Google Form with fields for
``First Name,'' ``Last Name'', and ``Contribution,'' and links the
results to a Google Sheet.\footnote{Users can find a template at
  \url{https://docs.google.com/forms/d/e/1FAIpQLSfElzd25aOTXDi8lNSdlfeWufRjhNqhE4jrPg7aHbhcQqP0PA/viewform?usp=sf_link}}
In the game, students may contribute any value (of points, candy, or
hypothetical dollars) between zero and ten. If an instructor chooses to
use class points for the payoffs, they may choose to endow students with
ten points for playing the game, or for higher stakes they may opt to
tell students that their contributions will come from their personal
point total they have already earned. Contributing from their earned
points makes the game more exciting, while endowing the students with
free points leads to less resentment both towards the professors and
among members of the class (even though the net effect on grades might
be the same!).

Once the instructor collects the contributions using the Google Form,
the application uses the results to calculate the returns (default =
20\%), and distributes the contributions and the returns equally among
all students.

\hypertarget{pollution-policies}{%
\subsection{Pollution Policies}\label{pollution-policies}}

The pollution game asks students to behave as profit-maximizing firms
that can produce up to two units of output at a price of four, but where
each unit of production produces a unit of pollution that creates a
negative externality of three. Students then choose how much of the
pollution from their production to abate (installing scrubbers, using
cleaner production methods). It then compares their choices under four
different pollution-abatement policies: (1) no regulation; (2)
command-and-control regulation; (3) a pollution tax equal to the extent
of the externality; and (4) a cap-and-trade market.

As with all previous activities, the instructor prepares a Google Form
with a corresponding linked Google Sheet for students to submit their
responses and to export the responses.\footnote{Template here:
  \url{https://docs.google.com/forms/d/e/1FAIpQLSew1-iIvA_2cMyzNNSlrhlcxrs2hpy2f_eJYyAHjBPVjg-7lA/viewform?usp=sf_link}.}
The template includes five sections. The setup section includes fields
for students to enter their (1) first name; (2) last name; (3)
randomly-assigned private cost of abating their first ``unit'' of
pollution (the lower of two integers between 1 and 6);\footnote{Template
  here:
  \url{https://docs.google.com/spreadsheets/d/1shTH39y65gaaYmevt4S6vtw1i-06hWEGeE1cPXe6J8M/edit?usp=sharing}.}
(4) cost of abating the second ``unit'' of pollution (the higher
integer). Encourage students to write down their abatement costs.

The section corresponding to no regulation asks students to choose their
profit-maximizing quantity under no regulation

\hypertarget{lobbying}{%
\subsection{Lobbying}\label{lobbying}}

When discussing monopoly, some instructors may find it useful to
introduce the concept of special interests. For example, one way that a
monopolist might create entry barriers would be to lobby the government
for a license to produce as a legal monopoly, or for an exclusive
contract to privately provide goods or services for the government. One
way to simulate this is through a lottery auction.

In this activity I typically endow students with five ``extra credit''
points to use for the activity. Alternatively, students could play with
candy or hypothetical dollars. They use this endowment to submit
political contributions to a politician (me) for the chance to win a
monopoly license worth four points (or pieces of candy or hypothetical
dollars). Each contribution represents one entry in a lottery for the
license and costs one point (or candy or hypothetical dollar).
Electronically, students submit their contributions by submitting a
Google Form\footnote{Template here:
  \url{https://docs.google.com/forms/d/e/1FAIpQLSdzMzydVZJrB2d57ZwQCkCPJcxNXogFHZTOOtZS75DHYvb5dg/viewform?usp=sf_link}.}
with only their first and last name \emph{the number of times
corresponding to the number of contributions they want to submit}. The
application randomly selects the winning row from the results
sheet\footnote{Template here:
  \url{https://docs.google.com/spreadsheets/d/1suPdRvwS_oM66cgZwx2nNw51XybEIbe9L7zkBB4dQOU/edit?usp=sharing}.}
(using the random seed ``8675309'' so that the result is replicable).

If instructors want to simulate ``campaign finance reform'' they can
make the contributions transparent by auctioning off the license in an
all-pay auction. Beware! The result of this auction can lead to students
bidding more than the license is ultimately worth!

\hypertarget{behavioral-economics}{%
\subsection{Behavioral Economics}\label{behavioral-economics}}

Numerous experiments describe different sorts of behavioral concepts in
economics and can be adapted to a classroom setting. The package
described here automates results for two of these, namely an anchoring
game based on Gelman and Glickman
(\protect\hyperlink{ref-gelman2000}{2000}) and the ultimatum game based
on Kahneman, Knetsch, and Thaler
(\protect\hyperlink{ref-kahneman1986}{1986}).

\hypertarget{anchoring-game}{%
\subsubsection{Anchoring Game}\label{anchoring-game}}

The anchoring game we automate here is based on the activity described
in Gelman and Glickman (\protect\hyperlink{ref-gelman2000}{2000}), which
begins by assigning students a random number (10 or 65) and responde to
the following:\\
1. What number were you assigned?/ 2. Thinking about that number, do you
think that the percentage of the world's countries that is located on
the continent of Africa is \emph{higher} or \emph{lower} than the number
you received?\\
3. What is your guess for the percentage of the world's countries that
is located in Africa?\\

Students submit their responses (without looking up the answers!) using
a Google Form,\footnote{Template here:
  \url{https://docs.google.com/forms/d/e/1FAIpQLSdzMzydVZJrB2d57ZwQCkCPJcxNXogFHZTOOtZS75DHYvb5dg/viewform?usp=sf_link}.}
which the instructor exports to a Google Sheet\footnote{Template here:
  \url{https://docs.google.com/spreadsheets/d/1suPdRvwS_oM66cgZwx2nNw51XybEIbe9L7zkBB4dQOU/edit?usp=sharing}.}
that the application uses to tabulate the results. The results include
the list of responses, a box plot, and test statistics comparing the two
randomly-assigned groups. ``Rational'' guessing should fail to reject
the hypothesis of no difference since which random (uninformative)
number a student receives should not influence rational guessers. In
reasonably large classes, we can usually reject this null hypothesis.

\hypertarget{ultimatum-game}{%
\subsubsection{Ultimatum Game}\label{ultimatum-game}}

The ultimatum game is a very popular game in experimental behavioral
economics. Students work in pairs, and determine which partner will be
the ``proposer'' and which will be the ``responder.'' The proposer
receives five points (or candies or tokens, or hypothetical dollars),
from which they offer a portion to the responder. The responder then
takes the option to accept or reject the offer. If the responder
accepts, then each partner receives benefits according to the agreement;
if the responder rejects, then both parties get nothing. The instructor
should encourage students to decide quickly and that this is not a
back-and-forth negotiation: take it or leave it offers only!

Each pair submits one version of the Google Form, which should include
fields for ``Proposer First Name'', ``Proposer Last Name'', ``Responder
First Name'', ``Responder Last Name'', ``Offer'', and ``Response'',
which the instructor prepares before class.\footnote{Template here:
  \url{https://docs.google.com/forms/d/e/1FAIpQLSeVGBjXI49-q0Tsj5iYAWe6UbsIBwzIryYh6Wz-rkQjZm7u9g/viewform?usp=sf_link}.}
The instructor should also link the Forms responses to a Google
Sheet\footnote{Template here:
  \url{https://docs.google.com/spreadsheets/d/12XRuR9HzJeshRM5npRILkXvbZbIxyJsj9abihVjYRpc/edit?usp=sharing}.}
that the package will use to tabulate the results, including a bar plot
of the offers, the list of responses, and the final tally of benefits
(e.g.~points) for each participant.

\hypertarget{the-econgame-package}{%
\section{\texorpdfstring{The \texttt{econGame}
Package}{The econGame Package}}\label{the-econgame-package}}

The \texttt{econGame} package pulls activity data from the Google Sheet
file generated for the form responses. It then tabulates, scores, and
graphs the responses. Users can download the package from
\url{https://github.com/bangecon/econGame}. To install the package, use
the \texttt{remotes} package and the \texttt{install\_github()}
function:

\texttt{remotes::install\_github("bangecon/econGame")}

Alternatively, if users have trouble using \texttt{remotes}, they try
using the \texttt{pkg\_install()} function from the \texttt{pak} package
(which sometimes works - albeit slightly less quickly - if
\texttt{remotes::install\_github()} gives an error) if they experience
difficulties installing \texttt{econgame} using
\texttt{remotes::install\_github}.

\hypertarget{exporting-the-results}{%
\subsection{Exporting the Results}\label{exporting-the-results}}

Once the students have completed each round of a game, the instructor
should pause to tabulate the results for that round. This is especially
the case in the entry and exit game, where convergence to the long-run
equilibrium requires students to know which sector earned economic
profits in the previous round. To do this, the instructor needs to click
the Google Sheets icon in the responses tab of the edit page of the
form. Figure 1 demonstrates where to find these tools in your Google
Form using the pit market trading equilibrium game as an example.

\begin{figure}
\centering
\includegraphics{images/Figure1.png}
\caption{Exporting Results to Google Sheets}
\end{figure}

This creates a Google Sheet that contains all of the data for tabulating
the results.\footnote{Link the Form to a Sheet to collect responses by
  selecting the ``Responses'' tab, and clicking ``Link to Sheets.''}

In order to link to your results, you will need to copy to your
clipboard either (a) the sharing url, which you can find by selecting
the ``Share'' link to the right of the menu bar; or (b) the sheet ID
from the URL (see Figure 2).

\begin{figure}
\centering
\includegraphics{images/Figure2.png}
\caption{Finding the Sheet ID}
\end{figure}

\hypertarget{tabulating-the-results}{%
\subsection{Tabulating the Results}\label{tabulating-the-results}}

Each game has its own function to tabulate its results, and for each
function there are slightly different options that the instructor can
adjust based on any changes they might have made to the default
parameters for the exercise. The only argument the instructor needs to
provide (and that does not have a preset default) is the ID of the
Google Sheet. The syntax and outputs for the different functions are:

\begin{itemize}
\item
  Each of the games shares the following arguments and outputs:

  \begin{itemize}
  \item
    Arguments

    \begin{itemize}
    \tightlist
    \item
      \texttt{sheet} (required): character string corresponding to the
      Google Sheets location (url or ID key) containing the individual
      submissions;
    \item
      \texttt{auth}: logical indicating whether to use an authentication
      token to access the Sheet containing the individual submissions;
    \item
      \texttt{names}: character list of the column names in
      \texttt{sheet};
    \item
      \texttt{email}: email address that matches the user account
      containing the Sheet with the individual submissions;
    \end{itemize}
  \item
    Outputs:

    \begin{itemize}
    \tightlist
    \item
      \texttt{type}: character for the type of activity (equlibriumGame,
      entryGame, etc.);
    \item
      \texttt{results}: table of original submissions (with market price
      and points added);
    \item
      \texttt{grades}: table aggregated points ``won'' by each student
      for all rounds of the activity;
    \end{itemize}
  \end{itemize}
\item
  Pit trading market equilibrium game-specific outputs:

  \begin{itemize}
  \tightlist
  \item
    \texttt{rounds}: numeric for the number of rounds in
    \texttt{results};
  \item
    \texttt{schedules}: list containing the supply and demand schedules
    for each round;
  \item
    \texttt{equilibria}: list containing the equilibria for each round;
  \end{itemize}
\item
  Entry-exit game-specific outputs:

\begin{verbatim}
-   `rounds`: numeric for the number of rounds in `results`;
-   `equilibria`: list containing the equilibria for each round;
\end{verbatim}
\item
  Bertrand game-specific arguments and outputs:

  \begin{itemize}
  \item
    Arguments:

    \begin{itemize}
    \tightlist
    \item
      \texttt{partners}: character equal to `random' or `students'
      indicating whether the function should use the
      \texttt{econGame::randomGroups()} function to randomly assign
      partners, or students will pick the partners themselves (default
      is `random');
    \item
      \texttt{a}: numeric value of the intercept of the inverse-demand
      function (default is 10);
    \item
      \texttt{b}: numeric value of the slope of the inverse-demand
      function (default is -1);
    \item
      \texttt{c}: numeric value of the firm's marginal cost (default is
      6);
    \item
      \texttt{f}: numeric value of the firm's fixed cost (default is 0);
    \end{itemize}
  \item
    Outputs:

    \begin{itemize}
    \tightlist
    \item
      \texttt{payoff}: table containing the matrix of payoffs for the
      reduced strategy combinations (fully collude or fully defect for
      each player);
    \item
      \texttt{output}: table containing the matrix of quantities
      corresponding to each combination of strategies;
    \item
      \texttt{price}: table containing the matrix of prices
      corresponding to the total market output for each combination of
      strategies.
    \end{itemize}
  \end{itemize}
\item
  Cournot game-specific arguments and outputs:

  \begin{itemize}
  \item
    Arguments:

    \begin{itemize}
    \tightlist
    \item
      \texttt{partners}: character equal to `random' or `students'
      indicating whether the function should use the
      \texttt{econGame::randomGroups()} function to randomly assign
      partners, or students will pick the partners themselves (default
      is `random');
    \item
      \texttt{a}: numeric value of the intercept of the inverse-demand
      function (default is 10).
    \item
      \texttt{b}: numeric value of the slope of the inverse-demand
      function (default is -1).
    \item
      \texttt{c}: numeric value of the firm's marginal cost (default is
      6).
    \item
      \texttt{f}: numeric value of the firm's fixed cost (default is 0).
    \end{itemize}
  \item
    Outputs:

    \begin{itemize}
    \tightlist
    \item
      \texttt{payoff}: table containing the matrix of payoffs for the
      reduced strategy combinations (collude or defect for each player);
    \item
      \texttt{output}: table containing the matrix of quantities
      corresponding to each combination of strategies;
    \item
      \texttt{price}: table containing the matrix of prices
      corresponding to the total market output for each combination of
      strategies.
    \end{itemize}
  \end{itemize}
\item
  Stackelberg game-specifirc arguments and outputs:

  \begin{itemize}
  \item
    Arguments:

    \begin{itemize}
    \tightlist
    \item
      \texttt{partners}: character equal to `random' or `students'
      indicating whether the function should use the
      \texttt{econGame::randomGroups()} function to randomly assign
      partners, or students will pick the partners themselves (default
      is `students');
    \item
      \texttt{a}: numeric value of the intercept of the inverse-demand
      function (default is 10).
    \item
      \texttt{b}: numeric value of the slope of the inverse-demand
      function (default is -1).
    \item
      \texttt{c}: numeric value of the firm's marginal cost (default is
      6).
    \item
      \texttt{f}: numeric value of the firm's fixed cost (default is 0).
    \end{itemize}
  \item
    Outputs:

    \begin{itemize}
    \tightlist
    \item
      \texttt{type} returns the type of activity (equlibriumGame).
    \item
      \texttt{results} returns the original submissions (with market
      price and points added).
    \item
      \texttt{grades} returns the aggregated points ``won'' by each
      student for the entire activity.
    \end{itemize}
  \end{itemize}
\end{itemize}

Another feature of the package is the ability to plot the results. The
syntax to plot any of the games described in this paper is simply
\texttt{plot(econGame,\ ...)}, where the (sole) argument is the name of
an object assigned by one of the \texttt{econGame} functions. The plot
the function generates depends on the type of game:

\begin{itemize}
\tightlist
\item
  For
  \texttt{type\ =\ \textquotesingle{}equilibriumGame\textquotesingle{}},
  \texttt{type\ =\ \textquotesingle{}entryGame\textquotesingle{}}, or
  \texttt{type\ =\ \textquotesingle{}pollutionGame\textquotesingle{}},
  plot the supply and demand functions with the corresponding equlibrium
  point.
\item
  For
  \texttt{type\ =\ \textquotesingle{}bertrandGame\textquotesingle{}},
  \texttt{type\ =\ \textquotesingle{}cournotGame\textquotesingle{}},
  \texttt{type\ =\ \textquotesingle{}stackelbergGame\textquotesingle{}},
  \texttt{type\ =\ \textquotesingle{}publicgoodGame\textquotesingle{}},
  or
  \texttt{type\ =\ \textquotesingle{}ultimatumGame\textquotesingle{}},
  plot a histogram of the strategy outcomes.
\item
  For
  \texttt{type\ =\ \textquotesingle{}anchoringGame\textquotesingle{}},
  plot a box plot of the guesses by anchor group.
\end{itemize}

\hypertarget{shiny-user-interface}{%
\subsection{Shiny User Interface}\label{shiny-user-interface}}

The functions for directly summarizing the results of the games may be
useful for tabulating the results for the purposes of awarding points to
the students who participated, but may not be the most
visually-appealing way to present the results in class. To improve the
user interface for a ``prettier'' presentation of the results, I have
built a Shiny Application UI for each of the games. In this interface,
the instructor can display the raw results, plots, or schedules of the
outcomes of the results by inputting the sheet ID (and other parameters)
in the input boxes, and by switching the display tabs of the results.

To run the Shiny App for a given package, I have written a function that
executes the app from the
\texttt{\textquotesingle{}\textasciitilde{}/inst/shiny-examples\textquotesingle{}}
folder of the package source. Once the instructor has installed the
package, all they need to do to execute the app is type one of the
following commands:
\texttt{\textquotesingle{}runEquilibriumGameApp()\textquotesingle{}},
\texttt{\textquotesingle{}runEntryGameApp()\textquotesingle{}},
\texttt{\textquotesingle{}runBertrandGameApp()\textquotesingle{}},
\texttt{\textquotesingle{}runCournotGameApp()\textquotesingle{}}, or
\texttt{\textquotesingle{}runStackelbergGameApp()\textquotesingle{}}.
When the instructor initiates the app, it will point the function to a
blank Google Sheet as the default, which will result in either blank
output, or an error. In order to read the submission results, the
instructor will need to paste the Google Sheet ID depicted in Figure 2.

Figure 3 shows the Shiny interface for \texttt{equilibriumGame}. Within
the output panel, the instructor (and students) will be able to see
tabbed output for different results, plots, and possibly the grade
outcomes if the instructor wishes to show it.

\begin{figure}
\centering
\includegraphics{images/Figure3.png}
\caption{Shiny User Interface}
\end{figure}

\hypertarget{discussion}{%
\section{Discussion}\label{discussion}}

This work provided examples of how to implement some simple market games
using R and Shiny. The objective of the functions developed for this
discussion was to help instructors adopt more creative and engaging
teaching methods in their principles classrooms by lowering both the
financial and time costs of tabulating the results and presenting
appealing summaries of the outcomes.

All of the examples can be implemented for the students using Google
Forms, which collects and exports the results in Google Sheets.
Instructors may then tabulate the results using the \texttt{econGame}
package via the HTML Shiny App with a single-function command in R. All
of these applications and packages come at no cost to the student. The
only potential cost barrier is the device - which could be as little as
a mobile device - each student would need in order to input their
responses.

I have platformed the package on GitHub
(\url{https://github.com/bangecon/econGame}), which allows users to make
pull requests to suggest edits and changes to improve the existing
experiments or suggest additions to the examples in the package. I also
plan to develop more functions and applications for the
\texttt{econGame} package from existing classroom experiments, which may
include a few original games. If you use these resources, please let me
know and share them with your colleagues.

\hypertarget{references}{%
\section*{References}\label{references}}
\addcontentsline{toc}{section}{References}

\hypertarget{refs}{}
\begin{CSLReferences}{1}{0}
\leavevmode\vadjust pre{\hypertarget{ref-anderson2000}{}}%
Anderson, Lisa R., and Sarah L. Stafford. 2000. {``Choosing Winners and
Losers in a Classroom Permit Trading Game.''} \emph{Southern Economic
Journal} 67 (1): 212--19. \url{https://doi.org/10.2307/1061622}.

\leavevmode\vadjust pre{\hypertarget{ref-brauer_games_2001}{}}%
Brauer, Jurgen, and Greg Delemeester. 2001. {``Games Economists Play:
{A} Survey of Non-Computerized Classroom-Games for College Economics.''}
\emph{Journal of Economic Surveys} 15 (2): 221--36.

\leavevmode\vadjust pre{\hypertarget{ref-chamberlin_experimental_1948}{}}%
Chamberlin, Edward H. 1948. {``An Experimental Imperfect Market.''}
\emph{Journal of Political Economy} 56 (2): 95--108.

\leavevmode\vadjust pre{\hypertarget{ref-cheung_using_2008}{}}%
Cheung, Stephen L. 2008. {``Using {Mobile} {Phone} {Messaging} as a
{Response} {Medium} in {Classroom} {Experiments}.''} \emph{The Journal
of Economic Education} 39 (1): 51--67.
\url{https://doi.org/10.3200/JECE.39.1.51-67}.

\leavevmode\vadjust pre{\hypertarget{ref-deyoung_market_1993}{}}%
DeYoung, Robert. 1993. {``Market Experiments: {The} Laboratory Versus
the Classroom.''} \emph{The Journal of Economic Education} 24 (4):
335--51.

\leavevmode\vadjust pre{\hypertarget{ref-dickie_classroom_2006}{}}%
Dickie, Mark. 2006. {``Do {Classroom} {Experiments} {Increase}
{Learning} in {Introductory} {Microeconomics}?''} \emph{The Journal of
Economic Education} 37 (3): 267--88.
\url{https://doi.org/10.3200/JECE.37.3.267-288}.

\leavevmode\vadjust pre{\hypertarget{ref-emerson_comparing_2004}{}}%
Emerson, Tisha LN, and Beck A. Taylor. 2004. {``Comparing Student
Achievement Across Experimental and Lecture-Oriented Sections of a
Principles of Microeconomics Course.''} \emph{Southern Economic
Journal}, 672--93.

\leavevmode\vadjust pre{\hypertarget{ref-garratt_free_2000}{}}%
Garratt, Rod. 2000. {``A Free Entry and Exit Experiment.''}
\emph{Journal of Economic Education} 31 (3): 237.
\url{https://www.proquest.com/docview/235237213/citation/8DF4FB2251DE443BPQ/1}.

\leavevmode\vadjust pre{\hypertarget{ref-gelman2000}{}}%
Gelman, Andrew, and Mark E. Glickman. 2000. {``Some Class-Participation
Demonstrations for Introductory Probability and Statistics.''}
\emph{Journal of Educational and Behavioral Statistics} 25 (1): 84--100.
\url{https://doi.org/10.3102/10769986025001084}.

\leavevmode\vadjust pre{\hypertarget{ref-gremmen_assessing_1997}{}}%
Gremmen, Hans, and Jan Potters. 1997. {``Assessing the Efficacy of
Gaming in Economic Education.''} \emph{The Journal of Economic
Education} 28 (4): 291--303.

\leavevmode\vadjust pre{\hypertarget{ref-holt_industrial_1993}{}}%
Holt, Charles A. 1993. {``Industrial {Organization}: {A} {Survey} of
{Laboratory} {Research}.''}

\leavevmode\vadjust pre{\hypertarget{ref-holt_classroom_1996}{}}%
---------. 1996. {``Classroom Games: {Trading} in a Pit Market.''}
\emph{Journal of Economic Perspectives} 10 (1): 193--203.

\leavevmode\vadjust pre{\hypertarget{ref-holt1997}{}}%
Holt, Charles A., and Susan K. Laury. 1997. {``Classroom Games:
Voluntary Provision of a Public Good.''} \emph{Journal of Economic
Perspectives} 11 (4): 209--15.
\url{https://doi.org/10.1257/jep.11.4.209}.

\leavevmode\vadjust pre{\hypertarget{ref-kahneman1986}{}}%
Kahneman, Daniel, Jack L. Knetsch, and Richard H. Thaler. 1986.
{``Fairness and the Assumptions of Economics.''} \emph{The Journal of
Business} 59 (4): S285--300. \url{https://www.jstor.org/stable/2352761}.

\leavevmode\vadjust pre{\hypertarget{ref-leuthold1993}{}}%
Leuthold, Jane H. 1993. {``A Free Rider Experiment for the Large
Class.''} \emph{The Journal of Economic Education} 24 (4): 353--63.
\url{https://doi.org/10.1080/00220485.1993.10844805}.

\leavevmode\vadjust pre{\hypertarget{ref-smith_experimental_1962}{}}%
Smith, Vernon L. 1962. {``An Experimental Study of Competitive Market
Behavior.''} \emph{Journal of Political Economy} 70 (2): 111--37.

\leavevmode\vadjust pre{\hypertarget{ref-stodder_experimental_1998}{}}%
Stodder, James. 1998. {``Experimental Moralities: {Ethics} in Classroom
Experiments.''} \emph{The Journal of Economic Education} 29 (2):
127--38.

\leavevmode\vadjust pre{\hypertarget{ref-watts_little_2008}{}}%
Watts, Michael, and William E. Becker. 2008. {``A {Little} {More} Than
{Chalk} and {Talk}: {Results} from a {Third} {National} {Survey} of
{Teaching} {Methods} in {Undergraduate} {Economics} {Courses}.''}
\emph{The Journal of Economic Education} 39 (3): 273--86.
\url{https://doi.org/10.3200/JECE.39.3.273-286}.

\leavevmode\vadjust pre{\hypertarget{ref-williams_economic_1993}{}}%
Williams, Arlington W., and James M. Walker. 1993. {``Economic
{Instruction} {Computerized} {Laboratory} {Exercises} for
{Microeconomics} {Education}: {Three} {Applications} {Motivated} by
{Experimental} {Economics}.''} \emph{Journal of Economic Education
(1986-1998)} 24 (4): 291.
\url{https://www.proquest.com/docview/216514673/abstract/C674FBF2453242F6PQ/1}.

\end{CSLReferences}

\end{document}
